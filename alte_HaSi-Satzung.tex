\documentclass[12pt]{satzung}

\usepackage[ngerman]{babel}
\usepackage{a4}
\usepackage{fontspec}
\usepackage{xunicode}
\usepackage{xltxtra}
\usepackage{enumerate}
\usepackage{fancyhdr}
\pagestyle{fancy}
\usepackage{lastpage}
\chead{~}
\lhead{~}
\rhead{~}
\cfoot{\thepage\ / \pageref{LastPage}}

\linespread{1.2}
\setmainfont{Vollkorn}

\setlength{\parindent}{0em}
\setlength{\parskip}{.5ex}

\title{Satzung des Vereins „HaSi“ (Hackerspace Siegen)}
\date{21.~02.~2012}

\begin{document}
\begin{center}
\includegraphics[width=7cm]{hasi-stempel}\\
\vspace{6mm}
{\large Satzung des Vereins}\\
\vspace{3mm}
{\Large\bf „HaSi“}\\
\vspace{2mm}
{\normalsize (Hackerspace Siegen)}\\
\vspace{4mm}
{\small Fassung vom~\makeatletter\@date\makeatother }
\end{center}

\vspace{8mm}

\setcounter{tocdepth}{4}
\tableofcontents

\newpage

\paragraph{Name, Sitz und Geschäftsjahr}
\label{sec:name-sitz-geschaeftsjahr}

\begin{enumerate}[(1)]
\item Der Name des Vereins lautet \emph{HaSi}

\item Der Verein hat seinen Sitz in Siegen, Bundesrepublik Deutschland. Sofern keine feste Geschäftsstelle eingerichtet ist, folgt die Verwaltung dem Wohnort des jeweiligen Vorstandsmitglieds, das die Geschäftsführung wahrnimmt.

\item Es wird eine Eintragung in das Vereinsregister des Amtsgerichtes Siegen angestrebt. 

% Bemerkung: Obwohl das Finanzamt keine Eintragung in das
% Vereinsregister benötigt um Gemeinnützigkeit zu erklären und uns
% auch vor finanziellen Belastung (zum Notar gehen, jedes Mal, wenn
% der Vorstand sich ändert) warnt, glauben wir, dass wir so mehr
% Spenden einsammeln können.

\item Das Geschäftsjahr ist das Kalenderjahr. 

\end{enumerate}


\paragraph{Ziele und Aufgaben des Vereins}
\label{sec:ziele-des-vereins}

\begin{enumerate}[(1)]


b) Zur Erreichung dieses Zieles sieht der Verein ihre Aufgabe darin, den Mitgliedern in der Aufführung Plattdeutscher Bühnenstücke zu beraten, Fortbildungsmaßnahmen und Öffentlichkeitsarbeit im Theaterbereich durch Angebote zu fördern, unter Ausschluss jeglichen privaten Erwerbsinteresses. Anteil am Gewinn steht weder dem Vorstand noch den Mitgliedern zu.


    \item Ziel und Zweck des Vereins "tollMut-Theater e.V." ist die Förderung von Kunst und Kultur.
    \item Der Verein erreicht seine Ziele insbesondere durch
    \begin{enumerate}[a)]
      \item gemeinschaftliche Entwicklung von Software sowie
        elektronischen Schaltungen, 
      \item Information der Öffentlichkeit,
      \item Anleitung zum kritischen Umgang mit elektronischen Medien,
      Software und Hardware
    \end{enumerate}
\end{enumerate}



\paragraph{Steuerbegünstigung}
\label{sec:steuerbeguenstigung}

\begin{enumerate}[(1)]
\item Der Verein verfolgt im Rahmen seiner Tätigkeit gemäß
\ref{sec:ziele-des-vereins} der Satzung ausschließlich und unmittelbar
gemeinnützige Zwecke im Sinne des Abschnittes steuerbegünstigte Zwecke
der Abgabenordnung (§§ 51ff. AO). Er ist selbstlos tätig und verfolgt
nicht in erster Linie eigenwirtschaftliche Zwecke.

\item Die Mittel des Vereins sind ausschließlich zu
satzungsgemäßen Zwecken zu verwenden. Die Mitglieder erhalten
ausschließlich Erstattungen entstandener Kosten, aber keine direkten
Zuwendungen aus Mitteln des Verein.

\item Niemand darf durch Vereinsausgaben, die dem
Vereinszweck fremd sind oder durch unverhältnismäßig hohe Vergütungen
begünstigt werden. Für den Ersatz von Aufwendungen ist, soweit nicht
andere gesetzliche Bestimmungen anzuwenden sind, das
Bundesreisekostengesetz maßgebend.
\end{enumerate}


\paragraph{Mitgliedschaft}
\label{sec:mitgliedschaft}

\begin{enumerate}[(1)]
    \item Mitglied kann jede natürliche Person werden, die das 14. Lebensjahr vollendet hat, sowie die Ziele des Vereins unterstützt.
    \item Die Mitgliedschaft wird durch eine schriftliche Beitrittserklärung beantragt. Über die Aufnahme entscheidet der Vorstand.
    \item Die Mitgliedschaft endet durch Austritt, Ausschluss oder Tod. Der Austritt erfolgt durch schriftliche Erklärung gegenüber dem Vorstand unter Einhaltung einer Frist von 4~Wochen. Die Beitragspflicht für das laufende Geschäftsjahr bleibt hiervon unberührt.
    \item Ein Mitglied kann durch Beschluss einer Mitgliederversammlung ausgeschlossen werden, wenn es den Vereinszielen zuwider handelt oder seinen Verpflichtungen gegenüber dem Verein nicht nachkommt.
    \item Die Beitragspflicht wird durch die Beitragsordnung geregelt.
    \item Bei Ausscheiden eines Mitgliedes aus dem Verein oder bei Vereinsauflösung besteht kein Anspruch auf Rückerstattung etwa eingebrachter Vermögenswerte. 

\end{enumerate}

\paragraph{Organe}
\label{sec:organe}

Organe des Vereins sind
\begin{enumerate}[a)]
    \item Mitgliederversammlung
    \item Vorstand 
\end{enumerate}

\paragraph{Mitgliederversammlung}
\label{sec:mitgliederversammlung}

\begin{enumerate}[(1)]
    \item Oberstes Organ ist die Mitgliederversammlung.
    \item Die Mitgliederversammlung stellt die Richtlinien für die Arbeit des Vereins auf und entscheidet Fragen von grundsätzlicher Bedeutung. Zu den Aufgaben der Mitgliederversammlung gehören insbesondere:
      \begin{enumerate}[a)]
      \item Bestimmung der Vorstandsmitglieder und Kassenprüfer durch geheime Wahl
      \item Diskussion über die bisherige und zukünftige Arbeit des Vereins
      \item Genehmigung des vom Vorstand vorgelegten Wirtschafts- und Investitionsplans
      \item Beschlussfassung über den Jahresabschluss
      \item Entgegennehmen des Geschäftsberichtes des Vorstandes
      \item Beschlussfassung über die Entlastung des Vorstandes
      \item Erlass der Beitragsordnung, die nicht Bestandteil der Satzung ist
      \item Beschlussfassung über die Übernahme neuer Aufgaben oder den Rückzug aus Aufgaben seitens des Vereins
      \item Beschlussfassung über Änderungen der Satzung und die Auflösung des Vereins 
      \end{enumerate}
    \item Die Einladung zur Mitgliederversammlung erfolgt durch den Vorstand in Textform. Die Einladungsfrist beträgt vier Wochen.
    \item Die Mitgliederversammlung tagt mindestens einmal im Jahr.
    \item Auf Antrag wenigstens eines Viertels der Mitglieder oder auf Beschluss des Vorstandes ist durch den Vorstand binnen sechs Wochen eine außerordentliche Mitgliederversammlung einzuberufen.
    \item Die Mitgliederversammlung ist beschlussfähig, wenn mehr als die Hälfte der Mitglieder anwesend ist; Falls dies nicht der Fall sein sollte, wird die Einladung erneut ausgesprochen. Die daraufhin einberufene Mitgliederversammlung ist dann in jedem Fall beschlussfähig.
    \item Über die Versammlung ist mindestens ein schriftliches Ergebnisprotokoll anzufertigen. Das Protokoll wird vom Protokollführer und dem Versammlungsleiter unterschrieben.
    \item Die Mitgliederversammlung fasst ihre Beschlüsse grundsätzlich mit einfacher Mehrheit in offener Abstimmung. Auschluss von Mitgliedern, Beschlüsse über Satzungsänderungen und Auflösung des Vereins erfordern mindestens doppelt so viele gültige Ja-Stimmen wie Nein-Stimmen. Auf Wunsch eines oder mehrerer stimmberechtigter Mitglieder ist geheim abzustimmen. 
\end{enumerate}

\paragraph{Vorstand}
\label{sec:vorstand}

\begin{enumerate}[(1)]
    \item Der Vorstand besteht aus dem Vorsitzenden, dem stellvertretenden Vorsitzenden und dem Schatzmeister. Sie bilden den Vorstand im Sinne von §26~BGB. Die Vorstandsmitglieder sind ehrenamtlich tätig.
    \item Zur rechtsverbindlichen Vertretung genügt die gemeinsame Zeichnung durch zwei Mitglieder des Vorstandes.
    \item Die Amtszeit der Vorstandsmitglieder beträgt 2~Jahre. Sie bleiben bis zur Bestellung des neuen Vorstandes kommissarisch im Amt.
    \item Die Beschlüsse sind schriftlich zu protokollieren und von dem Vorstandsvorsitzenden zu unterzeichnen.
    \item Der Vorstand gibt sich selbst eine Geschäftsordnung. 

\end{enumerate}

\paragraph{Satzungsänderung und Auflösung}
\label{sec:satzunsgaenderung-aufloesung}

\begin{enumerate}[(1)]
    \item Über Satzungsänderungen und die Auflösung des Vereins entscheidet die Mitgliederversammlung.
    \item Änderungen oder Ergänzungen der Satzung, die von der zuständigen Registerbehörde oder vom Finanzamt vorgeschrieben werden, werden vom Vorstand umgesetzt und bedürfen keiner Beschlussfassung durch die Mitgliederversammlung. Sie sind den Mitgliedern unverzüglich mitzuteilen.
    \item Bei Auflösung, bei Entziehung der Rechtsfähigkeit des Vereins oder bei Wegfall der steuerbegünstigten Zwecke fällt das gesamte Vermögen an Deutsche UNESCO-Kommission~e.\,V., die es ausschließlich und unmittelbar für gemeinnützige Zwecke zu verwenden hat. Sollte dieser Verein zu diesem Zeitpunkt nicht mehr gemeinnützig sein, fällt das Vermögen an eine andere von der Mitgliederversammlung zu bestimmende gemeinnützige Körperschaft, die das Vermögen für gemeinnützige Zwecke zu verwenden hat. 
\end{enumerate}


\paragraph*{Die Gründungsmitglieder}


\vspace{10mm}

\parbox[t]{.45\linewidth}{
\unterschrift{Tobias Becker}
\unterschrift{Florian Boor}
\unterschrift{Martin Braun}
\unterschrift{Simon Budig}
\unterschrift{Daniel Busch}
\unterschrift{Markus Dreiner}
\unterschrift{Dr. Silke Dreiner}
\unterschrift{Nils Faerber}
\unterschrift{Dr. Lars Fischer}}
\hfill
\parbox[t]{.45\linewidth}{
\unterschrift{Heinz Karsten Hiekmann}
\unterschrift{Frederik Lauber}
\unterschrift{Martin Reimann}
\unterschrift{Ole Reinhardt}
\unterschrift{Tim Nicolas Shirley}
\unterschrift{Peter Sponholtz}
\unterschrift{Sabine Vierbücher}
\unterschrift{Daniel Wagener}}

\end{document}

%%% Local Variables: 
%%% mode: latex
%%% TeX-master: t
%%% End: 

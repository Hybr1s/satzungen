\documentclass[a4paper,12pt]{scrartcl}
\usepackage{xltxtra}
\usepackage{unicode-math}
\usepackage[ngerman]{babel}
\usepackage[a4paper, top=30mm, left=30mm, right=30mm, bottom=30mm, headsep=10mm, footskip=12mm]{geometry} 
\usepackage{nameref}


\deffootnote{1em}{1em}{\textsuperscript{\thefootnotemark\ }}
\setcounter{tocdepth}{3}
\setcounter{secnumdepth}{3}
\usepackage{graphicx}
\usepackage{setspace} 
\usepackage{capt-of}
\usepackage{subfig}
\usepackage{fancyhdr}
\usepackage{url}

\usepackage[backend=biber,  citestyle=verbose-ibid]{biblatex}


%-------------------
%--Eigene Commands--
%-------------------
\newcommand{\bs}{\ensuremath{\backslash}}
\newcommand{\ko}[1]{}
\newcommand{\lz}[3]{\begin{singlespace} \begin{quotation}\vspace{-0.5cm}\glqq #1\grqq \footnote{Siehe \cite[#3.]{#2}} \end{quotation} \end{singlespace} }
\newcommand{\kz}[3]{\glqq #1\grqq \footnote{Siehe \cite[#3.]{#2}}}
\newcommand{\vw}[2]{\footnote{Vgl. \cite[#2.]{#1}}}

%--------------------------------------
%--------------------------------------
%--------------------------------------
%Ende des Kopfbereiches
%--------------------------------------
%--------------------------------------
%--------------------------------------


\begin{document}

% Kopf- und Fusszeile
\renewcommand{\sectionmark}[1]{\markright{#1}}
\renewcommand{\leftmark}{\rightmark}
\pagestyle{fancy}
\lhead{}
\chead{}
\rhead{\thesection\space\contentsname}
\renewcommand{\headrulewidth}{0.4pt}

% Vorspann
\renewcommand{\thesection}{\Roman{section}}
\renewcommand{\thesection}{\Roman{section}}
\pagenumbering{Roman}

%-------------------
%------Titelseite-----
%-------------------


\begin{titlepage}


\begin{minipage}[t]{0.76\textwidth}
\begin{flushleft}
\begin{small}

\end{small}
\end{flushleft}
\end{minipage}
\begin{minipage}[t]{0.24\textwidth}
\begin{flushleft}
\ \\
\today
\end{flushleft}
\end{minipage}

\vfill
\makebox[\textwidth]{\includegraphics[width=10cm]{tollmut.png}}
\vspace{0.5cm}

\begin{center}
\begin{Huge}

\textbf{Satzung}
\vspace{0.5cm}
\begin{small}
\begin{center}tollMut-Theater e.V.\end{center}
\end{small}

\end{Huge}
\end{center}

\vfill

\begin{center}
www.tollmut-theater.de
\end{center}  
\end{titlepage}

\newpage
%-------------------
%Inhaltsverzeichnis
%-------------------

\tableofcontents
\thispagestyle{empty}
\clearpage

%-------------------
%Kopfzeilen-Layout
%-------------------


% Kopfzeile
\renewcommand{\sectionmark}[1]{\markright{#1}}
\renewcommand{\subsectionmark}[1]{}
\renewcommand{\subsubsectionmark}[1]{}
\lhead{Satzung tollMut-Theater e.V.}
\rhead{\today}

\onehalfspacing
\renewcommand{\thesection}{\arabic{section}}
\renewcommand{\thesection}{\arabic{section}}
\setcounter{section}{0}
\pagenumbering{arabic}
\setcounter{page}{1}


\section*{\S{} 1 Name und Sitz}
\addcontentsline{toc}{section}{\S{} 1 Name und Sitz} 
\begin{description} 

\item[(1)] Der Verein führt den Namen „tollMut-Theater“. \item[(2)] Der Verein soll in das Vereinsregister eingetragen werden und führt danach den Zusatz e.V.
\item[(3)] Sein Sitz ist in Siegen, Bundesrepublik Deutschland. Sofern keine feste Geschäftsstelle eingerichtet ist, folgt die Verwaltung dem Wohnort des jeweiligen Vorstandsmitglieds, das die Geschäftsführung wahrnimmt. 
\item[(4)] Das Geschäftsjahr ist das Kalenderjahr.

\end{description}



\section*{\S{} 2 Zweck des Vereins}
\addcontentsline{toc}{section}{\S{} 2 Zweck des Vereins} 
\begin{description} 

\item[(1)] Der Zweck des Vereins ist die Förderung von Kunst und Kultur, insbesondere des freien Theaterspiels.
\item[(2)] Der Satzungszweck wird insbesondere verwirklicht durch die öffentliche Aufführung von Theaterstücken, sowie der Organisation sonstiger kultureller Veranstaltungen und Unternehmungen aller Art und des damit verbundenen Rahmenprogramms.
\item[(3)] Der Verein strebt die Zusammenarbeit mit anderen Organisationen und Einrichtungen an die ähnliche Ziele verfolgen oder auf dem oben genannten Gebiet tätig sind.
 
\end{description}



\section*{\S{} 3 Steuerbegünstigung}
\addcontentsline{toc}{section}{\S{} 3 Steuerbegünstigung}
\begin{description} 

\item[(1)] Der Verein verfolgt im Rahmen seiner Tätigkeit gemäß §2 der Satzung ausschließlich und unmittelbar gemeinnützige Zwecke im Sinne des Abschnittes steuerbegünstigte Zwecke der Abgabenordnung (§§ 51ff. AO). Er ist selbstlos tätig und verfolgt in erster Linie keine eigenwirtschaftlichen Zwecke.
\item[(2)] Die Mittel des Vereins sind ausschließlich zu satzungsgemäßen Zwecken zu verwenden. Die Mitglieder erhalten ausschließlich Erstattungen entstandener Kosten, aber keine direkten Zuwendungen aus Mitteln des Verein.
\item[(3)] Niemand darf durch Vereinsausgaben, die dem Vereinszweck fremd sind oder durch unverhältnismäßig hohe Vergütungen begünstigt werden. Für den Ersatz von Aufwendungen ist, soweit nicht andere gesetzliche Bestimmungen anzuwenden sind, das Bundesreisekostengesetz maßgebend.

\end{description}



\section*{\S{} 4 Mitgliedschaft}
\addcontentsline{toc}{section}{\S{} 4 Mitgliedschaft}
\begin{description} 

\item[(1)] Der Verein besteht auf aktiven Mitgliedern und Fördermitgliedern.
\item[(2)] Aktives Mitglied kann jede natürliche Person werden, die im Verein oder bei einem Projekt des Vereins aktiv mitwirken möchte.
\item[(3)] Fördermitglied kann jede natürliche oder juristische Person werden, die die Ziele und den Zweck des Vereins fördern und unterstützen möchte, ohne sich selbst aktiv am üblichen Vereinsgeschehen zu beteiligen.
\item[(4)] Der Beitritt zum Verein ist schriftlich zu erklären. Bei Minderjährigen ist der Aufnahmeantrag durch die gesetzlichen Vertreter zu stellen. Über die Aufnahme entscheidet der Vorstand.
\item[(5)] Die Beitragspflicht wird durch die Beitragsordnung geregelt.
\item[(6)] Die Mitgliedschaft als aktives Mitglied, sowie als Fördermitglied ist nicht übertragbar. 
\item[(7)] Die Mitgliedschaft endet durch freiwilligen Austritt, Ausschluss, Tod des Mitglieds oder Verlust der Rechtsfähigkeit bei juristischen Personen. 
\item[(8)] Der freiwillige Austritt erfolgt durch schriftliche Erklärung gegenüber dem Vorstand. Er ist außerhalb einer laufenden Produktion zum Quartalsende möglich und muss mindestens einen Monat vorher schriftlich erklärt werden. Aktiv an einer laufenden Produktion Mitwirkende ist der Austritt erst zum Quartalsende nach Ende der Produktion möglich. Auch hier gilt eine einmonatige Kündigungsfrist. Die Dauer der Produktion bezeichnet den Zeitraum von der Besetzung eines Projekts bis zum Tag nach dessen letzter Aufführung.
\item[(9)] Ein Mitglied kann aus dem Verein mit sofortiger Wirkung ausgeschlossen werden, wenn das Mitglied gegen die Satzung, Ordnungen, den Satzungszweck oder die Interessen des Vereins verstößt. Über den Ausschluss eines Mitglieds entscheidet der Vorstand.
\item[(10)] Nach dem Ende der Mitgliedschaft, gleich aus welchem Grund, erlöschen alle Ansprüche aus dem Mitgliedsverhältnis. Eine Rückgewähr von Beiträgen, Spenden oder sonstigen Unterstützungsleistungen ist grundsätzlich ausgeschlossen. Der Anspruch des Vereins auf eventuell rückständige Beitragsforderungen bleibt hiervon unberührt.

\end{description}



\section*{\S{} 5 Rechte und Pflichten der Mitglieder}
\addcontentsline{toc}{section}{\S{} 5 Rechte und Pflichten der Mitglieder}
\begin{description} 

\item[(1)] Die Mitglieder haben das Recht, gegenüber dem Vorstand und der Mitgliederversammlung Anträge zu stellen.
\item[(2)] Aktive Mitglieder besitzen das aktive und passive Wahlrecht, sowie das Antrags-, Stimm- und Rederecht auf Mitgliedsversammlungen.
\item[(3)] Fördermitglieder besitzen das Rede- und Antragsrecht auf Versammlungen, jedoch kein Stimm- oder Wahlrecht.
\item[(4)] Die Vereinsmitglieder sind zur Zahlung des in der Beitragsordnung geregelten Mitgliedsbeitrages verpflichtet. Sie sind außerdem dazu verpflichtet, dem Verein Änderungen ihrer Postadresse, E-Mail-Adresse und Bankverbindung umgehend mitzuteilen. Änderung bezüglich der Bankeinzüge können nur bei Übermittlung 3 Wochen vor der Buchung berücksichtigt werden. Für Folgen, die sich daraus ergeben, dass das Mitglied dieser Pflicht nicht nachkommt, haftet das Mitglied und stellt den Verein von jeglicher Haftung frei.

\end{description}



\section*{\S{} 6 Organe des Vereins}
\addcontentsline{toc}{section}{\S{} 6 Organe des Vereins}
\begin{description} 

\item[(1)] Der Vorstand
\item[(2)] Die Mitgliederversammlung

\end{description}



\section*{\S{} 7 Der Vorstand}
\addcontentsline{toc}{section}{\S{} 7 Der Vorstand}
\begin{description} 

\item[(1)] Der Vorstand wird im Sinne des §26 BGB aus einem Mitglied als Person für den Vorsitz gebildet.
\item[(2)] Die Person für den Vorsitz wird von der Mitgliederversammlung für eine unbefristete Amtszeit mit relativer Mehrheit gewählt und bleibt bis zur Eintragung einer anderen Person für den Vorsitz kommissarisch im Amt.
\item[(3)] Der Vorsitz ernennt unter Berücksichtigung der Vorschläge der Mitgliederversammlung ein Mitglied für die Kassenführung auf unbestimmte Zeit.
\item[(4)] Die künstlerische Leitung wird in der Regel vom Vorstand für die Dauer eines Projektes ernannt und kann in Ausnahmefällen durch den Vorstand auch abgesetzt werden.
\item[(5)] Die künstlerische Leitung obliegt die Regieführung und Leitung eigener künstlerischer Projekte wie zum Beispiel Inszenierungen, Aufführungen oder Performances des Vereins. Er besitzt das Vorschlagsrecht für die Auswahl von Stücken und entscheidet mit dem Vorstand über die Annahme von Stücken.
\item[(6)] Der Vorstand gibt sich selbst eine Geschäftsordnung.
\item[(7)] Zur rechtsverbindlichen Vertretung genügt die Zeichnung durch ein Mitglied des Vorstandes.
\item[(8)] Der Vorstand entscheidet über die Übernahme neuer Aufgaben und/oder den Rückzug aus Aufgaben seitens des Vereins.
\item[(9)] Die Kassenführung und künstlerische Leitung sind in ihrem Aufgabenfeld und im Sinne des Vorstandes zeichnungsberechtigt.
\item[(10)] Der Vorstand, die Kassenführung wie auch die künstlerische Leitung dürfen für ihre Tätigkeit eine angemessene Vergütung erhalten, gemäß der Vergütungsvorgabe des Finanzamtes. Das Nähere beschließt die Mitgliederversammlung.
\item[(11)] Der Vorstand, die Kassenführung wie auch die künstlerische Leitung haften nicht für leichte Fahrlässigkeit.

\end{description}


\section*{\S{} 8 Die Mitgliederversammlung}
\addcontentsline{toc}{section}{\S{} 8 Die Mitgliederversammlung}
\begin{description} 

\item[(1)] Oberstes Organ ist die Mitgliederversammlung.
\item[(2)] Die Mitgliederversammlung ist mindestens ein Mal pro Jahr.
\item[(3)] Die Einladung erfolgt mit einer mindestens 14-tägigen Frist durch den Vorstand in Textform.
\item[(4)] Auf Antrag von wenigstens 5 der stimmberechtigten Mitglieder oder auf Beschluss des Vorstandes, ist durch den Vorstand binnen sechs Wochen eine außerordentliche Mitgliederversammlung einzuberufen. Die Einladungsfrist beträgt in diesem Fall mindestens 7 Tage.
\item[(5)] Zu den Aufgaben der Mitgliederversammlung gehören insbesondere:
\begin{description} 
\item[(a)] Bestimmung des Vorstands und Kassenprüfer durch geheime Wahl
\item[(b)] Diskussion über die bisherige und zukünftige Arbeit des Vereins
\item[(c)] Genehmigung des vom Vorstand vorgelegten Wirtschafts- und Investitionsplans
\item[(d)] Beschlussfassung über den Jahresabschluss
\item[(e)] Entgegennehmen des Geschäftsberichtes des Vorstandes
\item[(f)] Beschlussfassung über die Entlastung des Vorstandes
\item[(g)] Erlass der Beitragsordnung, die nicht Bestandteil der Satzung ist
\item[(h)] Beschluss über die Vergütung des Vorstandes, der Kassenführung und der künstlerischen Leitung
\item[(i)] Beschlussfassung über Änderungen der Satzung und die Auflösung des Vereins 
\end{description}
\item[(6)] Jedes stimmberechtigte Vereinsmitglied hat das gleiche Stimmgewicht.
\item[(7)] Die Mitgliederversammlung ist beschlussfähig, wenn mindestens 5 stimmberechtigte Mitglieder anwesend sind. Falls dies nicht der Fall sein sollte, wird direkt ein neuer Termin bestimmt, ohne die Einladungsfrist. Die daraufhin einberufene Mitgliederversammlung ist dann in jedem Fall beschlussfähig.
\item[(8)] Über die Versammlung ist mindestens ein schriftliches Ergebnisprotokoll anzufertigen. Das Protokoll ist vom Versammlungsleiter und/oder Schriftführer zu unterschrieben.
\item[(9)] Die Mitgliederversammlung fasst ihre Beschlüsse grundsätzlich mit relativer Mehrheit in offener Abstimmung. Ausschluss von Mitgliedern, Beschlüsse über Satzungsänderungen und Auflösung des Vereins erfordern mindestens doppelt so viele gültige Ja-Stimmen wie Nein-Stimmen. Auf Wunsch eines oder mehrerer stimmberechtigter Mitglieder ist geheim abzustimmen.

\end{description}


\section*{\S{} 9 Satzungsänderung und Auflösung}
\addcontentsline{toc}{section}{\S{} 9 Satzungsänderung und Auflösung}
\begin{description} 

\item[(1)] Über Satzungsänderungen und die Auflösung des Vereins entscheidet die Mitgliederversammlung.
\item[(2)] Änderungen oder Ergänzungen der Satzung, die von der zuständigen Registerbehörde oder vom Finanzamt vorgeschrieben werden, werden vom Vorstand umgesetzt und bedürfen keiner Beschlussfassung durch die Mitgliederversammlung. Sie sind den Mitgliedern unverzüglich mitzuteilen.
\item[(3)] Bei Auflösung, bei Entziehung der Rechtsfähigkeit des Vereins oder bei Wegfall der steuerbegünstigten Zwecke fällt das gesamte Vermögen an den "Mehr GRIPS! - Förderer des GRIPS Theaters e.V.", die es ausschließlich und unmittelbar für gemeinnützige Zwecke zu verwenden hat. Sollte dieser Verein zu diesem Zeitpunkt nicht mehr gemeinnützig sein, fällt das Vermögen an eine andere von der Mitgliederversammlung zu bestimmende gemeinnützige Körperschaft, die das Vermögen für gemeinnützige Zwecke zu verwenden hat.

\end{description}


\end{document}